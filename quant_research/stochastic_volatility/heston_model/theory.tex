\documentclass[11pt,letterpaper]{article}

% Packages
\usepackage[utf8]{inputenc}
\usepackage[margin=1in]{geometry}
\usepackage{amsmath,amssymb,amsthm}
\usepackage{graphicx}
\usepackage{hyperref}

% Document info
\title{\textbf{Heston Model: Theory and Applications}}
\author{Quantitative Research}
\date{\today}

\begin{document}

\maketitle
\tableofcontents
\newpage

\section{Key Fundamentals}

The Heston model (1993) is a \textbf{stochastic volatility model} that extends the Black-Scholes framework by allowing volatility itself to be random rather than constant. It's one of the most widely used models in quantitative finance for pricing derivatives and managing risk. While Black-Scholes assumes constant volatility, the Heston model recognizes that volatility fluctuates over time in response to market conditions, making it far more realistic for capturing the dynamics of financial markets.

\subsection{Core Mathematical Structure}

The model is defined by two coupled stochastic differential equations (SDEs) that jointly describe the evolution of asset prices and their volatility:

\textbf{Asset Price Process:}
\begin{equation}
dS_t = \mu S_t \, dt + \sqrt{v_t} \, S_t \, dW_t^S
\end{equation}

This equation describes how the stock price $S_t$ evolves over time. The first term ($\mu S_t dt$) represents the deterministic drift or expected return, while the second term ($\sqrt{v_t} S_t dW_t^S$) captures the random fluctuations. Critically, the magnitude of these fluctuations is determined by $\sqrt{v_t}$, which varies stochastically.

\textbf{Variance Process (Cox-Ingersoll-Ross process):}
\begin{equation}
dv_t = \kappa(\theta - v_t)dt + \sigma\sqrt{v_t} \, dW_t^v
\end{equation}

This equation governs the evolution of variance (volatility squared). The term $\kappa(\theta - v_t)$ creates mean reversion: when variance is above its long-term mean $\theta$, the drift pulls it downward, and vice versa. The term $\sigma\sqrt{v_t} dW_t^v$ adds randomness to the variance itself, creating the ``volatility of volatility'' effect.

\noindent \textbf{Parameter Definitions:}
\begin{itemize}
    \item $S_t$ = asset price at time $t$ (the observable market price)
    \item $v_t$ = instantaneous variance at time $t$ (volatility squared, not directly observable)
    \item $\mu$ = drift rate or expected return (annualized, typically calibrated to risk-neutral measure)
    \item $\kappa$ = mean reversion speed (controls how quickly variance reverts; higher values = faster reversion). Typical range: 0.5 to 5.0
    \item $\theta$ = long-term mean variance level (the equilibrium variance; $\sqrt{\theta}$ is long-term volatility). Typical range: 0.01 to 0.09 (10\% to 30\% volatility)
    \item $\sigma$ = volatility of volatility (vol-of-vol; controls how much variance fluctuates). Typical range: 0.1 to 1.0
    \item $v_0$ = initial variance (starting value at $t=0$, often calibrated to current implied volatility)
    \item $W^S$ and $W^v$ = correlated Wiener processes (standard Brownian motions) with correlation $\rho$
    \item $\rho$ = correlation between asset returns and variance (typically -0.5 to -0.8 for equities, capturing the leverage effect)
\end{itemize}

\subsection{The Feller Condition}

A critical constraint is the \textbf{Feller condition}: 
\begin{equation}
2\kappa\theta \geq \sigma^2
\end{equation}

This ensures variance remains strictly positive (never goes negative), which is essential for mathematical consistency and realistic modeling.

\section{What It Models}

\subsection{Stochastic Volatility}

Unlike Black-Scholes (constant volatility), Heston captures the empirical observation that volatility changes randomly over time. Periods of market calm alternate with periods of high turbulence.

\subsection{Volatility Clustering}

The mean-reverting variance process (CIR) naturally produces \textbf{volatility clustering} -- the tendency for high-volatility periods to cluster together, and likewise for low-volatility periods. This matches real market behavior.

\subsection{Leverage Effect (Asymmetric Volatility)}

The correlation parameter $\rho$ (typically $-0.5$ to $-0.8$) captures the \textbf{leverage effect}: when stock prices fall, volatility tends to rise. This asymmetry is crucial for modeling equity markets and produces realistic volatility smiles.

\subsection{Volatility Smile/Skew}

The Heston model produces the characteristic \textbf{volatility smile} and \textbf{skew} observed in options markets -- out-of-the-money puts are more expensive than Black-Scholes predicts, while deep out-of-the-money calls are cheaper.

\subsection{Semi-Closed Form Solutions}

One of Heston's key advantages is that European option prices can be computed via \textbf{Fourier inversion} and the characteristic function:

\begin{equation}
C(S_0, v_0, K, T) = S_0 P_1 - K e^{-rT} P_2
\end{equation}

where $P_1$ and $P_2$ are probabilities computed using characteristic functions, providing fast and accurate pricing without Monte Carlo simulation.

\section{What It Ignores}

\subsection{Jumps in Asset Prices}

The Heston model assumes continuous price paths. It doesn't capture sudden \textbf{discrete jumps} (like earnings announcements, geopolitical shocks, or market crashes). Extensions like Bates model add jump components.

\subsection{Time-Varying Parameters}

All parameters ($\kappa, \theta, \sigma, \rho$) are assumed constant over time. In reality, market dynamics evolve -- mean reversion strength and long-term volatility levels change with market regimes.

\subsection{Multiple Volatility Factors}

Heston uses a single stochastic volatility factor. Empirical research suggests multiple volatility components (short-term vs. long-term) may be needed for accurate modeling, especially for long-dated options.

\subsection{Interest Rate Risk}

The model assumes a constant risk-free rate. For long-dated derivatives or in volatile rate environments, stochastic interest rates (hybrid models) may be necessary.

\subsection{Microstructure Effects}

Transaction costs, bid-ask spreads, liquidity constraints, and market impact are ignored. The model assumes frictionless markets with continuous trading.

\subsection{Non-Normal Return Distributions}

While better than Black-Scholes, the model still doesn't fully capture extreme tail events (fat tails, extreme kurtosis) observed during market crises.

\section{Why People Use It}

\subsection{Realistic Volatility Dynamics}

The stochastic volatility feature captures real market behavior better than constant-volatility models, producing more accurate option prices across strikes and maturities.

\subsection{Volatility Smile Consistency}

Heston naturally generates the volatility smile/skew without ad-hoc adjustments. This makes it ideal for pricing exotic derivatives and managing volatility risk.

\subsection{Computational Efficiency}

The semi-closed form solution via characteristic functions makes Heston \textbf{much faster} than pure Monte Carlo methods while maintaining accuracy. Option prices can be computed in milliseconds.

\subsection{Calibration Stability}

With only 5 parameters ($\kappa, \theta, \sigma, \rho, v_0$), the model strikes a good balance between flexibility and parsimony. It can be calibrated to market data reasonably well without overfitting.

\subsection{Industry Standard}

Heston is a \textbf{benchmark model} in quantitative finance. It's widely implemented in trading systems, risk management platforms, and academic research. This standardization facilitates communication and comparison.

\subsection{Path-Dependent Products}

For path-dependent derivatives (Asian options, barrier options, variance swaps), Heston provides a tractable framework that can be simulated efficiently or sometimes solved semi-analytically.

\subsection{Volatility Trading}

The model is essential for \textbf{volatility arbitrage} and \textbf{variance swap} trading. The variance process directly relates to realized and implied volatility, allowing traders to take positions on volatility itself.

\subsection{Risk Management}

Financial institutions use Heston for:
\begin{itemize}
    \item \textbf{VaR (Value at Risk)} calculations with dynamic volatility
    \item \textbf{Stress testing} under various volatility scenarios
    \item \textbf{Hedging strategies} that account for changing volatility (vega risk)
\end{itemize}

\subsection{Model Extensions}

Heston serves as a foundation for more sophisticated models:
\begin{itemize}
    \item \textbf{Bates model}: Heston + jumps
    \item \textbf{Double Heston}: Two variance factors
    \item \textbf{Heston-Hull-White}: Stochastic volatility + stochastic interest rates
\end{itemize}

\section{Practical Considerations}

\subsection{When to Use Heston}
\begin{itemize}
    \item Pricing equity options (especially short-to-medium term)
    \item Volatility surface modeling and arbitrage
    \item When volatility risk is significant
    \item When you need balance between realism and computational speed
\end{itemize}

\subsection{When to Consider Alternatives}
\begin{itemize}
    \item \textbf{Local volatility models} (Dupire): Better calibration to vanilla options but poor for exotics
    \item \textbf{Jump-diffusion models}: When discontinuous price movements matter (earnings, crises)
    \item \textbf{SABR model}: Popular for interest rate derivatives
    \item \textbf{Rough volatility models}: Cutting-edge research for very high-frequency data
\end{itemize}

\section{Conclusion}

The Heston model represents a sweet spot in quantitative finance: sophisticated enough to capture key market realities (stochastic volatility, leverage effect, volatility smile), yet tractable enough for practical implementation. While it makes simplifying assumptions and ignores certain market features, its widespread adoption reflects its robust performance across a wide range of derivatives pricing and risk management applications. 

Understanding the Heston model is essential for any quantitative analyst working in derivatives markets. Its mathematical elegance combined with practical utility has made it the gold standard for stochastic volatility modeling over the past three decades.

\begin{thebibliography}{99}
\bibitem{heston1993}
Heston, S. L. (1993). ``A Closed-Form Solution for Options with Stochastic Volatility with Applications to Bond and Currency Options''. \textit{Review of Financial Studies}, 6(2), 327-343.

\bibitem{cox1985}
Cox, J. C., Ingersoll, J. E., \& Ross, S. A. (1985). ``A Theory of the Term Structure of Interest Rates''. \textit{Econometrica}, 53(2), 385-407.
\end{thebibliography}

\end{document}
