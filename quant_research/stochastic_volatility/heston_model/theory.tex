\documentclass[11pt,letterpaper]{article}

% Packages
\usepackage[utf8]{inputenc}
\usepackage[margin=1in]{geometry}
\usepackage{amsmath,amssymb,amsthm}
\usepackage{graphicx}
\usepackage{hyperref}

% Document info
\title{\textbf{Heston Model: Theory and Applications}}
\author{Quantitative Research}
\date{January 21, 2026}

\begin{document}

\maketitle
\tableofcontents
\newpage

\section{Key Fundamentals}

The Heston model (1993) is a \textbf{stochastic volatility model} that extends the Black-Scholes framework by allowing volatility itself to be random rather than constant. It's one of the most widely used models in quantitative finance for pricing derivatives and managing risk. While Black-Scholes assumes constant volatility, the Heston model recognizes that volatility fluctuates over time in response to market conditions, making it far more realistic for capturing the dynamics of financial markets.

\subsection{Core Mathematical Structure}

The model is defined by two coupled stochastic differential equations (SDEs) that jointly describe the evolution of asset prices and their volatility:

\textbf{Asset Price Process:}
\begin{equation}
dS_t = \mu S_t \, dt + \sqrt{v_t} \, S_t \, dW_t^S
\end{equation}

This equation describes how the stock price $S_t$ evolves over time. The first term ($\mu S_t dt$) represents the deterministic drift or expected return, while the second term ($\sqrt{v_t} S_t dW_t^S$) captures the random fluctuations. Critically, the magnitude of these fluctuations is determined by $\sqrt{v_t}$, which varies stochastically.

\textbf{Variance Process (Cox-Ingersoll-Ross process):}
\begin{equation}
dv_t = \kappa(\theta - v_t)dt + \sigma\sqrt{v_t} \, dW_t^v
\end{equation}

This equation governs the evolution of variance (volatility squared). The term $\kappa(\theta - v_t)$ creates mean reversion: when variance is above its long-term mean $\theta$, the drift pulls it downward, and vice versa. The term $\sigma\sqrt{v_t} dW_t^v$ adds randomness to the variance itself, creating the ``volatility of volatility'' effect.

\noindent \textbf{Parameter Definitions:}
\begin{itemize}
    \item $S_t$ = asset price at time $t$ (the observable market price)
    \item $v_t$ = instantaneous variance at time $t$ (volatility squared, not directly observable)
    \item $\mu$ = drift rate or expected return (annualized, typically calibrated to risk-neutral measure)
    \item $\kappa$ = mean reversion speed (controls how quickly variance reverts; higher values = faster reversion). Typical range: 0.5 to 5.0
    \item $\theta$ = long-term mean variance level (the equilibrium variance; $\sqrt{\theta}$ is long-term volatility). Typical range: 0.01 to 0.09 (10\% to 30\% volatility)
    \item $\sigma$ = volatility of volatility (vol-of-vol; controls how much variance fluctuates). Typical range: 0.1 to 1.0
    \item $v_0$ = initial variance (starting value at $t=0$, often calibrated to current implied volatility)
    \item $W^S$ and $W^v$ = correlated Wiener processes (standard Brownian motions) with correlation $\rho$
    \item $\rho$ = correlation between asset returns and variance (typically -0.5 to -0.8 for equities, capturing the leverage effect)
\end{itemize}

\subsection{The Feller Condition}

A critical constraint is the \textbf{Feller condition}: 
\begin{equation}
2\kappa\theta \geq \sigma^2
\end{equation}

This mathematical condition ensures that variance remains strictly positive and never reaches zero (or goes negative), which would be nonsensical in a financial context. The Feller condition states that the strength of mean reversion (measured by $2\kappa\theta$) must dominate the randomness in variance (measured by $\sigma^2$). 

When this condition holds, the variance process has sufficient ``pull'' toward its mean $\theta$ to prevent the Brownian motion from driving it to zero. In practical terms:
\begin{itemize}
    \item If $2\kappa\theta > \sigma^2$: Variance strictly positive for all time (mathematically guaranteed)
    \item If $2\kappa\theta = \sigma^2$: Variance can reach zero but doesn't become negative
    \item If $2\kappa\theta < \sigma^2$: Variance can become negative (invalid, violates model assumptions)
\end{itemize}

Violation of the Feller condition doesn't immediately break numerical simulations, but it creates theoretical inconsistencies and can lead to pathological behavior in extreme scenarios. Most calibrations to real market data satisfy this condition comfortably.

\section{What It Models}

\subsection{Stochastic Volatility}

Unlike Black-Scholes which assumes constant volatility (a single number for all time), Heston captures the empirical observation that volatility changes randomly over time following its own stochastic process. In real markets, we observe periods of market calm (low volatility) that can suddenly transition to periods of high turbulence (high volatility), such as during financial crises or major news events. 

The Heston model doesn't just allow volatility to change deterministically—it makes volatility itself a random variable with its own source of uncertainty. This second source of randomness (in addition to price randomness) is what makes it a ``two-factor'' model and fundamentally different from Black-Scholes.

\subsection{Volatility Clustering}

The mean-reverting variance process (CIR) naturally produces \textbf{volatility clustering}—the well-documented phenomenon where high-volatility periods tend to cluster together, and likewise for low-volatility periods. This was famously noted by Mandelbrot: ``large changes tend to be followed by large changes...and small changes tend to be followed by small changes.''

The mean reversion parameter $\kappa$ controls how long volatility tends to stay elevated or depressed before reverting to its long-term mean. A lower $\kappa$ means slower mean reversion and more persistent volatility clustering, while higher $\kappa$ means volatility quickly returns to normal after shocks. This clustering behavior is visible in realized volatility time series of virtually all financial assets and is a key feature that Black-Scholes cannot capture.

\subsection{Leverage Effect (Asymmetric Volatility)}

The correlation parameter $\rho$ (typically $-0.5$ to $-0.8$ for equity markets) captures the \textbf{leverage effect}: when stock prices fall, volatility tends to rise, and when prices rise, volatility tends to fall. This creates an asymmetric response.

The economic intuition comes from two sources:
\begin{itemize}
    \item \textbf{Financial leverage}: When stock prices fall, the debt-to-equity ratio of the firm increases (fixed debt, lower equity value), making the stock riskier and more volatile
    \item \textbf{Volatility feedback}: Bad news simultaneously lowers prices and increases uncertainty about future outcomes
\end{itemize}

This negative correlation is crucial for modeling equity derivatives correctly. It produces the characteristic ``skew'' in implied volatility surfaces where out-of-the-money puts (downside protection) are more expensive than Black-Scholes would predict, while out-of-the-money calls are cheaper. The Heston model's ability to generate this skew endogenously (without manual adjustment) is one of its most valuable features.

\subsection{Volatility Smile/Skew}

The Heston model produces the characteristic \textbf{volatility smile} and \textbf{skew} observed in options markets—patterns that Black-Scholes fundamentally cannot generate. In the Black-Scholes world, all options on the same underlying with the same maturity should have the same implied volatility. In reality, implied volatility varies with strike price, creating a ``smile'' or ``skew'' shape.

For equity markets, Heston naturally produces:
\begin{itemize}
    \item Higher implied volatility for out-of-the-money puts (lower strikes) — traders pay premium for downside protection
    \item Lower implied volatility at-the-money — the baseline volatility level
    \item Lower implied volatility for out-of-the-money calls (higher strikes) — less demand for extreme upside bets
\end{itemize}

This ``smirk'' or ``skew'' pattern reflects market participants' asymmetric fear of downside versus upside moves. The Heston model generates this naturally through the leverage effect ($\rho < 0$) and stochastic volatility, without requiring ad-hoc adjustments to volatility for different strikes.

\subsection{Semi-Closed Form Solutions}

One of Heston's key practical advantages is that European option prices can be computed via \textbf{Fourier inversion} and characteristic functions, avoiding computationally expensive Monte Carlo simulation. The call option price is given by:

\begin{equation}
C(S_0, v_0, K, T) = S_0 P_1 - K e^{-rT} P_2
\end{equation}

where $P_1$ and $P_2$ are risk-neutral probabilities computed using the characteristic function of the log-stock price. This characteristic function has a known closed form in the Heston model, making pricing extremely fast.

The computation involves numerical integration (typically Gauss-Legendre quadrature), but this is orders of magnitude faster than Monte Carlo methods and produces highly accurate prices in milliseconds. This efficiency is crucial for:
\begin{itemize}
    \item Real-time pricing of large portfolios
    \item Calibration to market data (requiring thousands of price evaluations)
    \item Risk management systems that need to reprice under multiple scenarios
    \item High-frequency trading applications
\end{itemize}

The semi-closed form solution strikes an ideal balance: more realistic than Black-Scholes, yet far faster than pure simulation methods.

\section{What It Ignores}

Despite its sophistication, the Heston model makes several simplifying assumptions that may not hold in extreme market conditions or for all asset classes.

\subsection{Jumps in Asset Prices}

The Heston model assumes \textbf{continuous price paths}—prices can only change smoothly over infinitesimally small time intervals. It doesn't capture sudden \textbf{discrete jumps} such as:
\begin{itemize}
    \item Overnight gaps from earnings announcements (stock suddenly opens 10\% higher/lower)
    \item Geopolitical shocks (Brexit vote, 9/11, COVID-19 announcement)
    \item Flash crashes or sudden liquidity events
    \item Central bank surprise rate decisions
\end{itemize}

These discontinuous moves are important for pricing short-dated options and out-of-the-money options where jump risk is significant. Extensions like the \textbf{Bates model} (1996) add a jump component to both price and variance processes, improving the fit to short-term options markets where jump fears are priced in.

\subsection{Time-Varying Parameters}

All five Heston parameters ($\kappa, \theta, \sigma, \rho, v_0$) are assumed \textbf{constant over time}. In reality:
\begin{itemize}
    \item Mean reversion speed $\kappa$ may change during crises (volatility becomes more persistent)
    \item Long-term mean $\theta$ shifts with market regimes (low-vol regime vs. high-vol regime)
    \item Correlation $\rho$ can vary (leverage effect stronger during bear markets)
    \item Vol-of-vol $\sigma$ may increase during periods of uncertainty
\end{itemize}

This assumption of time-homogeneity simplifies calibration but can lead to poor out-of-sample performance when market dynamics shift. Practitioners often recalibrate frequently (daily or weekly) to adapt to changing conditions, but this introduces model risk and calibration instability.

\subsection{Multiple Volatility Factors}

Heston uses a \textbf{single stochastic volatility factor}. Empirical research suggests that volatility has multiple timescales:
\begin{itemize}
    \item \textbf{Short-term component}: High-frequency volatility spikes that mean-revert quickly (days to weeks)
    \item \textbf{Long-term component}: Persistent volatility shifts that last months to years (regime changes)
\end{itemize}

A single-factor model cannot simultaneously capture both fast mean reversion for short-dated options and slow mean reversion for long-dated options. The \textbf{Double Heston model} addresses this by using two independent variance processes with different mean reversion speeds, improving the fit across the entire maturity spectrum but doubling the number of parameters.

\subsection{Interest Rate Risk}

The model assumes a \textbf{constant risk-free rate} $r$. For short-dated equity derivatives, this is often acceptable since interest rate changes have minimal impact. However, for:
\begin{itemize}
    \item Long-dated options (5+ years)
    \item Interest rate derivatives (swaptions, caps, floors)
    \item Currency derivatives where rates differ across currencies
    \item Markets with volatile interest rate environments
\end{itemize}

Constant rates become unrealistic. Hybrid models like \textbf{Heston-Hull-White} combine stochastic volatility for the underlying with stochastic interest rates, at the cost of significantly increased complexity.

\subsection{Microstructure Effects}

The model assumes \textbf{frictionless markets} with:
\begin{itemize}
    \item Zero transaction costs
    \item No bid-ask spreads
    \item Infinite liquidity (can trade any size without market impact)
    \item Continuous trading (no market closures or gaps)
    \item No margin requirements or funding costs
\end{itemize}

In practice, these frictions matter, especially for:
\begin{itemize}
    \item High-frequency delta hedging strategies (transaction costs accumulate)
    \item Large institutional positions (market impact and liquidity risk)
    \item Illiquid derivatives (wide bid-ask spreads)
    \item Overnight risk and weekend/holiday gaps
\end{itemize}

Incorporating realistic trading costs and liquidity constraints typically requires numerical methods or simulation-based approaches beyond the Heston framework.

\subsection{Non-Normal Return Distributions}

While Heston improves upon Black-Scholes by incorporating stochastic volatility, it still produces return distributions that are conditionally normal (given the volatility path). It doesn't fully capture:
\begin{itemize}
    \item \textbf{Fat tails}: Extreme events occur more frequently than normal distributions predict
    \item \textbf{Extreme kurtosis}: ``Black swan'' events with probabilities far exceeding Gaussian assumptions
    \item \textbf{Higher-order moments}: Skewness and kurtosis that vary systematically with market conditions
\end{itemize}

During the 2008 financial crisis, even sophisticated stochastic volatility models underestimated tail risk. Models incorporating jumps (Bates), time-varying jump intensity (SVCJ), or rough volatility better capture these extreme events, though at the cost of analytical tractability.

\section{Why People Use It}

\subsection{Realistic Volatility Dynamics}

The stochastic volatility feature captures real market behavior better than constant-volatility models, producing more accurate option prices across strikes and maturities.

\subsection{Volatility Smile Consistency}

Heston naturally generates the volatility smile/skew without ad-hoc adjustments. This makes it ideal for pricing exotic derivatives and managing volatility risk.

\subsection{Computational Efficiency}

The semi-closed form solution via characteristic functions makes Heston \textbf{much faster} than pure Monte Carlo methods while maintaining accuracy. Option prices can be computed in milliseconds.

\subsection{Calibration Stability}

With only 5 parameters ($\kappa, \theta, \sigma, \rho, v_0$), the model strikes a good balance between flexibility and parsimony. It can be calibrated to market data reasonably well without overfitting.

\subsection{Industry Standard}

Heston is a \textbf{benchmark model} in quantitative finance. It's widely implemented in trading systems, risk management platforms, and academic research. This standardization facilitates communication and comparison.

\subsection{Path-Dependent Products}

For path-dependent derivatives (Asian options, barrier options, variance swaps), Heston provides a tractable framework that can be simulated efficiently or sometimes solved semi-analytically.

\subsection{Volatility Trading}

The model is essential for \textbf{volatility arbitrage} and \textbf{variance swap} trading. The variance process directly relates to realized and implied volatility, allowing traders to take positions on volatility itself.

\subsection{Risk Management}

Financial institutions use Heston for:
\begin{itemize}
    \item \textbf{VaR (Value at Risk)} calculations with dynamic volatility
    \item \textbf{Stress testing} under various volatility scenarios
    \item \textbf{Hedging strategies} that account for changing volatility (vega risk)
\end{itemize}

\subsection{Model Extensions}

Heston serves as a foundation for more sophisticated models:
\begin{itemize}
    \item \textbf{Bates model}: Heston + jumps
    \item \textbf{Double Heston}: Two variance factors
    \item \textbf{Heston-Hull-White}: Stochastic volatility + stochastic interest rates
\end{itemize}

\section{Practical Considerations}

\subsection{When to Use Heston}
\begin{itemize}
    \item Pricing equity options (especially short-to-medium term)
    \item Volatility surface modeling and arbitrage
    \item When volatility risk is significant
    \item When you need balance between realism and computational speed
\end{itemize}

\subsection{When to Consider Alternatives}
\begin{itemize}
    \item \textbf{Local volatility models} (Dupire): Better calibration to vanilla options but poor for exotics
    \item \textbf{Jump-diffusion models}: When discontinuous price movements matter (earnings, crises)
    \item \textbf{SABR model}: Popular for interest rate derivatives
    \item \textbf{Rough volatility models}: Cutting-edge research for very high-frequency data
\end{itemize}

\section{Conclusion}

The Heston model represents a sweet spot in quantitative finance: sophisticated enough to capture key market realities (stochastic volatility, leverage effect, volatility smile), yet tractable enough for practical implementation. While it makes simplifying assumptions and ignores certain market features, its widespread adoption reflects its robust performance across a wide range of derivatives pricing and risk management applications. 

Understanding the Heston model is essential for any quantitative analyst working in derivatives markets. Its mathematical elegance combined with practical utility has made it the gold standard for stochastic volatility modeling over the past three decades.

\begin{thebibliography}{99}
\bibitem{heston1993}
Heston, S. L. (1993). ``A Closed-Form Solution for Options with Stochastic Volatility with Applications to Bond and Currency Options''. \textit{Review of Financial Studies}, 6(2), 327-343.

\bibitem{cox1985}
Cox, J. C., Ingersoll, J. E., \& Ross, S. A. (1985). ``A Theory of the Term Structure of Interest Rates''. \textit{Econometrica}, 53(2), 385-407.
\end{thebibliography}

\end{document}
